\documentclass{article}
\usepackage{graphicx}

\begin{document} 
\begin{center}
{\bf \Large  Homework 5} \\
(Due on: May 13, 9:00PM, by e-mail)
\end{center}

\begin{figure}[h]
\begin{center}
\includegraphics[width=0.6\textwidth]{Fig.pdf}
\end{center}
\end{figure}

\noindent The figure shows a robot moving left with the average velocity of $0.8cm/s$.
While moving, it measures each second ($\Delta T = 1s$) the distance from the obstacle
behind it (the corner) and the obstacle in front of it (the wall). In this scenario,
we consider the corner as the reference point, i.e., with the coordinate 0. 
In order to simultaneously estimate the robot position, as well
as the position of the wall, we use the state vector $\underline{x}$ with three components: 
$x_1$ is the roboti position measured from the corner, $x_2$ is the robot velocity with
positive direction towards the wall and $x_3$ is the wall position measured from
the corner.  $y_1$ is the measurement of the distance from the robot to the corner
and $y_2$ is the measurement of the distance from the robot to the wall. A linear model describing this scenario is
\begin{equation}
  \underline{x}(k+1)=\left[ 
  \begin{array}{ccc}
     1 & 1 & 0  \\
     0 & 0 & 0  \\
     0 & 0 & 1 
  \end{array}
   \right] \underline{x}(k)+
   \left[ 
  \begin{array}{c}
     0   \\
     0.8   \\
     0
  \end{array}
   \right]+ \left[
     \begin{array}{c}
     0   \\
     0.1   \\
     0
  \end{array}
   \right] w(k)
\end{equation}
where the Gaussian random variable $w(k)$ has zero mean and variance 1. 
The observation model is
\begin{equation}
 \underline{y}(k) =\left[
     \begin{array}{ccc}
     1 & 0 &  0  \\
     -1 & 0 & 1   
  \end{array}
   \right] \underline{x}(k)+\underline{\theta}(k)
\end{equation}
where the covariance matrix of Gaussian measurement noise $\underline{\theta}$ is $diag(10, 10)$. All variables in the above equations are expressed in $cm$ and $cm/s$. The course webpage provides you the sequence of the measurements $y_1$ and $y_2$ recorded for 
$k = 1, 2,... $ in the fille {\bf roboMes.mat}. The data are organized in such a way that $y(1, k)$ denotes the measurement $y_1(k)$ and $y(2, k)$ denotes the measurement $y_2(k)$, where $k=1,2,3...$

Design the Kalman filter that takes the measurement sequences and produces the estimation sequences of the robot position, its velocity and the distance between the wall and the corner.
\begin{itemize}
\item[a)] Plot the measurements on the same diagram. 
\item[b)] Plot the estimated robot position and the corresponding variance. 
\item[c)] Plot the estimated wall position and the corresponding variance. 
\end{itemize}
What is the precision in estimating the position of the robot and the wall ?
Compare the precision to the intensity of measurement errors. Comment your
results and trends in data.

\end{document}

