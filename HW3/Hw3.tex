\documentclass{article}
\usepackage{graphicx}

\begin{document} 
\begin{center}
{\bf \Large  Homework 3} \\
(Due on: Fri, April 22, 8:00PM, by e-mail)
\end{center}


\noindent {\bf Problem 1.} A dynamical system is governed by the stochastic difference equation 
\begin{equation}
  \underline{x}(k+1)=\left[
  \begin{array}{cc} 
    1.5 & 1 \\
   -0.7 & 0
  \end{array}
  \right] \underline{x}(k)+
  \left[ 
  \begin{array}{c}
   1.0 \\
   0.5 
  \end{array}
\right] w(k)
\end{equation}

\noindent where $w(k)$ is a sequence of independent Normal(mean=0,std=1) (Gaussian)
random variables. \\

\noindent a) Derive the expression for the state covariance and determine the state covariance in the steady state $P_{\infty}$. \\

\noindent b) Write a MATLAB code that generates $\underline{x}(k)$ for $k = 1..500$ and 
$\underline{x}(0) = [0\ 0]^T$. Generate five sequences of $\underline{x}(k)$ and plot them on two 
diagrams, one for $x_1(k)$ and one for $x_2(k)$.  \\

\noindent c) Generate 50 trajectories for $\underline{x}(k)$ and from them compute the 
standard deviation for $x_1(k)$ and $x_2(k)$. \\

\noindent d) Use the analytical results to compute the state $\underline{x}(k)$ covariance 
matrix $P(k)$ and corresponding standard deviations for $x_1(k)$ and $x_2(k)$. 
Compare them with the computed values  in point $c$. \\

\noindent {\bf Problem 2.} (only for graduate students) A 2D random walk is described by the following system of 
stochastic differential equations
\begin{eqnarray}
  dx&=&dw_1 \\
  dy&=&dw_2
\end{eqnarray}
where $dw_1$ and $dw_2$ are increments of two independent unit intensity Wiener processes. \\
 
\noindent (a) Use the initial condition $x(0)=y(0)=0$ to compute a trajectory $x(k\Delta T)$, $y(k \Delta T)$ for $\Delta T=0.1s$ and the final time point $T_f=10s$. Repeat the computations 
100 times and plot all trajectories in the $x$-$y$ coordinate system. \\

\noindent (b) For each trajectory, compute $r(k\Delta T)=\sqrt{x(k\Delta T)^2+y(k\Delta T)^2}$ 
and their average for each $k$, $\overline{r}(k\Delta T)$. Plot on the same diagram  
 $r(k\Delta T)$ versus the time in green, as well as  $\overline{r}(k\Delta T)$ in blue. \\

\noindent (c) What polynomial is the best fit to $\overline{r}(k\Delta T)$?

\end{document}