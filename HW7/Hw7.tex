\documentclass{article}
\usepackage{graphicx}

\begin{document} 
\begin{center}
{\bf \Large  Homework 7} \\
(Due on: May 27, 9:00PM)
\end{center}

The HW6 figure shows a robot moving left. While moving, it measures each second ($\Delta T$ = 1s) the distance from the obstacle behind it (the corner) and the obstacle
in front of it (the wall). In this scenario, we consider the corner as the reference
point, i.e., with the coordinate 0. As shown in class, in order to simultaneously
estimate the robot position and the position of the wall, we use the state vector
$\underline{x}$ with the following three components: $x_1$ is the robot position measured from
the corner, $x_2$ is the robot velocity with the positive direction towards the wall
and $x_3$ is the wall position measured from the corner. The measurement $y_1$ is
the distance from the robot to the corner and $y_2$ is the distance from the robot
to the wall. A linear model describing this scenario is
\begin{equation}
  \underline{x}(k+1)=\left[ 
  \begin{array}{ccc}
     1 & 1 & 0  \\
     0 & 1 & 0  \\
     0 & 0 & 1 
  \end{array}
   \right] \underline{x}(k)+ \left[
     \begin{array}{c}
     0   \\
     0.1   \\
     0
  \end{array}
   \right] w(k)
\end{equation}
where the Gaussian random variable w(k) has zero mean and variance 1. The
observation model is
\begin{equation}
 \underline{y}(k) =\left[
     \begin{array}{ccc}
     1 & 0 &  0  \\
     -1 & 0 & 1   
  \end{array}
   \right] \underline{x}(k)+\underline{\theta}(k)
\end{equation}
where the covariance matrix of Gaussian measurement noise $\underbar{$\theta$}$ 
is $diag(10, 10)$. All variables in the above equations are expressed in $cm$ and $cm=s$.

The course webpage (HW6) provides you the sequence of the measurements
$y_1$ and $y_2$ recorded for $k = 1, 2,...$ in the fille roboMes.mat. The data are
organized so that $y(1, k)$ denotes the measurement $y_1(k)$ and $y(2, k)$ 
denotes the measurement $y_2(k)$, where $k=1,2,3....$. 

Design the Kalman smoother that takes the measurement sequences and
produces the estimation sequences of the robot position, its velocity and the
distance between the wall and the corner given all available data. \\

a) Plot on the same diagram the result of the forward Kalman filter and
RTS backward iterations for {\bf the robot position}.

b) Plot on the same diagram the result of the forward Kalman filter and
RTS backward iterations for {\bf the robot velocity}.

d) Plot on the same diagram the result of the forward Kalman filter and
RTS backward iterations for {\bf the wall position}.

e) For the robot position, compare the variance resulting from the Kalman
filter and the variance resulting from the smoother.

Comment your results and trends in data. {\bf Note:} Please keep in mind that
you can use your Kalman filter code from HW6 with a slight modification, i.e.,
after removing 0.8 in the equation defining the velocity. Choose your initial
condition for $x_1(0)$ and $x_3(0)$, as well as corresponding variances the 
same way you did for the Kalman filter in HW6. {\bf However, for $x_2(0)$ use the 
initial condition $0.2$ or $1.6$ and the variance $1$.}

\end{document}