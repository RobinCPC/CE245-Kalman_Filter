\documentclass{article}
\usepackage{graphicx}

\begin{document} 
\begin{center}
{\bf \Large  Homework 1} \\
(Due on: April 6 by 8:00PM)
\end{center}

\noindent A simple nonlinear model of pendulum dynamics is given by:
\begin{equation}
  m L \ddot{x}=-mg \sin x
\end{equation}
This second order differential equation can be re-written as a system of 
differential equations
\begin{eqnarray}
 \dot{x}_1 &=& x_2 \\
 \dot{x}_2 &=& -\frac{g}{L} \sin x_1,\ \ \ \ \ g/L=0.1
\end{eqnarray}

\noindent a) Use MATLAB function ode45 (or ode 23) to numerically solve the system of differential equations for the initial condition $x_1=\pi/6$, $x_2=0$ and plot diagrams for $x_1(t)$, $x_2(t)$ and $x_1(t)$ vs. $x_2(t)$. \\

\noindent b) Let us consider the pendulum with a random acceleration term $f(t)$ 
\begin{eqnarray}
 \dot{x}_1 &=& x_2 \\
 \dot{x}_2 &=& -\frac{g}{L} \sin x_1 + f(t),\ \ \ \ \ g/L=0.1
\end{eqnarray}
where the random acceleration term $f(t)$ is described as follows: $t_1$, $t_2$, ...$t_k$ are time 
points of random events, such that the probability density function of time intervals
$t_k-t_{k-1}$  is 
\begin{equation}
  p(t_{k}-t_{k-1})=\lambda e^{-\lambda(t_k-t_{k-1})},\ \ \ \lambda=0.02,  t_k > t_{k-1}, t_0=0
\end{equation}
and $f(t)=\sum_{k} f(t_k)\delta(t-t_k)$, $t_k>0$
\begin{equation}
 f(t_k) = \left\{ 
  \begin{array}{ll}
    0,& t \ne t_k \\
    \varepsilon \sim \mathcal{N}(0,\sigma=0.2),& t = t_k 
  \end{array} \right.
\end{equation}  
Numerically solve (4) and (5) for the initial condition $x_1=0$, $x_2=0$ and plot diagrams 
for $x_1(t)$, $x_2(t)$ and $x_1(t)$ vs. $x_2(t)$.

\end{document}